
% JuliaCon proceedings template
\documentclass{juliacon}
\setcounter{page}{1}

\usepackage{cleveref}

% See https://tex.stackexchange.com/questions/550123/underscore-in-doi-in-bibtex-file
\usepackage{underscore}

\usepackage{tikz}
\usetikzlibrary{arrows,matrix,decorations.pathreplacing,positioning,chains,fit,shapes,calc,3d}
\tikzset{box/.style={rectangle, draw=black, fill=black!8, align=center, rounded corners}}
\tikzset{rep/.style={rectangle, draw=juliablue, fill=juliablue!8, align=center, rounded corners}}

\definecolor{juliablue}{rgb}{0.251, 0.388, 0.847}
\definecolor{juliagreen}{rgb}{0.22 , 0.596, 0.149}
\definecolor{juliapurple}{rgb}{0.584, 0.345, 0.698}
\definecolor{juliared}{rgb}{0.796, 0.235, 0.2  }

\newcommand{\Vrep}{V-rep}
\newcommand{\hrep}{H-rep}

\newcommand{\jlpkg}[1]{\texttt{#1.jl}}

\begin{document}

\input{header}

\maketitle

\begin{abstract}

    This paper introduces a Julia interface for polyhedral computation libraries such as cdd, lrs, qhull, porta or the parma polyhedra library.
    This notably includes conversion between the representation of a polyhedron as the intersection of half-spaces and hyperplanes and
    its representation as the convex hull of points and rays.
    Moreover, the interface covers canonicalization tools such as the removal of redundant elements from the representations or
    detection of hyperplanes formed by the intersection of half-spaces or lines by a conic combination of rays.

    The package also include a default pure-Julia library that implements the double description method for the representation conversion
    and canonicalization using a linear programming solver implementing \jlpkg{MathOptInterface}.
    A polyhedron object is provided in order to abstract away the differences between the libraries and hence allow writing library-agnostic code.
    This object lazily applies the representation conversion when needed.

\end{abstract}

\section{Introduction}

There are two dual approaches to represent a convex set.
First, it can be described as the Minkowski sum between the convex hull of points
and the conic hull of rays; this is the \emph{Vertex representation} or \emph{\Vrep{}}.
The dual alternative is to obtain the set as the intersection of half-spaces;
this is the \emph{Half-spaces representation} or \emph{h-representation}
It turns out that a convex set admits a finite \Vrep{} if and only if it admits a finite \hrep{}.
Such convex sets are called \emph{polyhedra} or even \emph{polytopes} when the \Vrep{} contains no ray.

\section{Representation conversion}

Several algorithms exist to convert a \hrep{} to a \Vrep{} and vice-versa.
This include the \emph{double description} algorithm~\cite{motzkin1953double,fukuda1996double} implemented in cdd~\cite{fukuda2003cdd},
the \emph{reversed search vertex enumeration} algorithm~\cite{avis1996reverse} implemented in lrs~\cite{avis1994ac,avis2000revised} and
the \emph{quick hull} algorithm implemented in qhull~\cite{barber1996quickhull}.
As detailed in \cite{avis1995good}, each of these algorithms outperforms the other two for specific families of polyhedra.
This highlights the importance of implementing algorithm relying on polyhedral computation in a library-agnostic way
so that each application can benefit from the appropriate library.

%\subsection{Arithmetic}

Representation conversion can be challenging to implement reliably using floating point arithmetic.
For this reason, lrs and cdd allow exact arithmetic using the equivalent of \texttt{Rational\{BigInt\}}.
The code of these libraries need to use the specific API of GMP~\cite{granlund1996gnu} in order to get the best performance.
Thanks to \jlpkg{MutableArithmetics}~\cite{Legat2024},
a default implementation of the double description algorithm could be implemented
in \jlpkg{Polyhedra} that is agnostic to the number type without sacrificing performance.

\section{Canonicalization}

There are two aspects to reducing the size of a representation.
First, elements can be removed if they are redundant.
Second, several rays (resp. half-spaces) can be combined
into a line (resp. hyperplane);
%this is referred to as the \emph{linearity} of the representation.
this is called \emph{linearity} detection~\cite{avis2002canonical}.
Detecting either redundancy or linearity can be
achieved via Linear Programming~\cite{avis2000revised}.
Both cdd and lrs incorporate linear programming solvers within
their library which can be used for these tasks.
\jlpkg{Polyhedra} can also use any other linear programming solver
implementing \jlpkg{MathOptInterface}~\cite{legat2021mathoptinterface}.

By using both the primal and dual result from the linear program,
the linearity detection implemented in \jlpkg{Polyhedra}
only solves a number of linear programs equal to the
number of lines (resp. hyperplanes) that are to be detected in the worst case.
This is an improvement over the implementation in cdd
that solves a number of linear programs equal to the number
of rays (resp. half-spaces) of the input representation in the worst case~\cite[Remark~1.3.1]{legat2020set}.

\section{Lazy conversion}

A typical polyhedral computation pipeline consists in creating a few polyhedra
either from their H or \Vrep{}, perform operations between them,
and then querying the H or \Vrep{} of the result.
As illustrated in \cref{op}, for some mathematical operations on polyhedra,
the use of either the V or \hrep{} is much more natural\footnote{%
  In view of the direct relation between the V (resp. H)-representation and the \emph{support} (resp \emph{gauge}) function
  of the polyhedron \cite[Proposition~A.30]{legat2023Chapter} (resp. \cite[Proposition~A.27]{legat2023Chapter}),
  this can understood from properties of the gauge and support functions summarized in \cite[Table~2]{legat2023Chapter}.
}.
%For instance, the intersection needs the \hrep{}
%while the Minkowski sum needs the \Vrep{}.
A polyhedron objects lazily performs the representation conversion
when needed.

However, this choice is not always straightforward.
To illustrate this, consider the problem of
checking the membership of a point in a polyhedron.
This can either be achieved by verifying its membership to
each half-space of the \hrep{} or by checking whether it
would be redundant if added to the \Vrep{} by solving a linear program.
While checking membership to a half-space is much faster
than solving a linear program,
the number of half-space can be much larger than the size of the linear program.
For instance, the $n$-dimensional cross-polytope (the polar of the hypercube)
is the intersection of $2^n$ nonredundant half-spaces but the convex hull
of only $2n$ points.
For this polytope, using the \Vrep{} is therefore more efficient than using the \hrep{} when $n$ is large enough.

Similarly, projecting a polyhedron is straightforward given a \Vrep{}
while for a \hrep{}, both Fourier-Motzkin~\cite[p.~155–156]{schrijver1986Theory}
and block elimination~\cite{fukuda2003cddlib} are computationally more costly.
However, as a $n$-dimensional hypercube is the convex hull of $2^n$ nonredundant points but
the intersection of only $2n$ half-spaces,
the projection of the hypercube is better done through the \hrep{} for large $n$.

In order to resolve this, a polyhedron has 3 possible states:
the \hrep{} has been computed,
the \Vrep{} has been computed or
both.
If only one is computed, it is used to check whether the point belongs to the polyhedron.
If both are computed then the \hrep{} is used for membership check and \Vrep{} for projection;
this preference is indicated by the green arrow in \cref{op}.
The user can then influence the choice of representation used
%for checking the membership of a point or projecting a polyhedron
by explicitly requesting the computation of such representation beforehand.

\begin{figure}
  \centering
  \begin{tikzpicture}[scale = 0.8, every node/.style={scale=1}]
    \node[rep] (H)     at (-2,  0) {H-rep};
    \node[rep] (V)     at ( 2,  0) {V-rep};
    \node[box] (INT)   at (-4,  1) {$\cap$};
    \draw[thick, ->] (H.north west) to (INT.south east);
    \node[box] (CHEBY) at (-4, -1) {\begin{minipage}{3em}\centering Cheby center\end{minipage}};
    \draw[color=juliapurple, thick, ->] (H.south west) to node[rotate=20, below] {\includegraphics[width=2.5em]{JuMP.png}} (CHEBY.north east);
    \node[box] (CART)  at (-2.5,  2) {\begin{minipage}{4em}\centering Cartesian product\end{minipage}};
    \draw[color = juliagreen, thick, ->] (H.north) to (CART.south);
    \draw[color = juliared, thick, dashed, ->] (V.north west) to (CART.south east);
    \node[box] (PLOT)  at (-1,  2) {Plot};
    \draw[thick, -] (H.north east) to (-1, 1);
    \draw[thick, -] (V.north west) to (-1, 1);
    \draw[thick, ->] (-1, 1) to (PLOT.south);
    \node[box] (VOL)   at ( 1,  2) {Volume};
    \draw[thick, -] (H.north east) to ( 1, 1);
    \draw[thick, -] (V.north west) to ( 1, 1);
    \draw[thick, ->] ( 1, 1) to (VOL.south);
    \node[box] (IN)    at (2.5,  2) {\texttt{in}};
    \draw[color=juliagreen, thick, ->] (H.north east) to (IN.south west);
    \draw[color=juliared, thick, -] (V.north) to (IN.south);
    \draw[color=juliapurple, thick, dashed, ->] (V.north) to node[rotate=75, below] {\includegraphics[width=2.5em]{JuMP.png}} (IN.south);
    \node[box] (PLUS)  at ( 4,  1) {$+$};
    \draw[dashed, thick, ->] (V.north east) to (PLUS.south west);
    \node[box] (CONVH) at ( 4, -1) {Convex hull};
    \draw[thick, ->] (V.south east) to (CONVH.north west);
    \node[box] (AM)    at ( 2, -2) {$A \times \cdot$};
    \draw[thick, ->] (V.south) to (AM.north);
    \node[box] (AB)    at (-2, -2) {$A \backslash \cdot$};
    \draw[thick, ->] (H.south) to (AB.north);
    \node[box] (PROJ)  at ( 0, -2) {Project};
    \draw[color = juliared, thick, dashed, ->] (H.south east) to (PROJ.north west);
    \draw[color = juliagreen, thick, ->] (V.south west) to (PROJ.north east);
    \node[box] (TR)    at ( 0,  0) {Translate};
    \draw[thick, ->] (H.east) to (TR.west);
    \draw[thick, ->] (V.west) to (TR.east);

    \draw[color = juliagreen, thick, ->]  (-4, -2.7) to (-3.5, -2.7) node[right] {Preferred};
    \draw[color = juliared, thick, ->]    (-4, -3.2) to (-3.5, -3.2) node[right] {Fallback if other representation not computed};
    \draw[color = juliapurple, thick, ->] (-4, -3.7) to (-3.5, -3.7) node[right] {Solve linear program using \includegraphics[height=1em]{jump.png}};
    \draw[dashed, thick, ->]               (-4, -4.2) to (-3.5, -4.2) node[right] {Size of resulting representation may grow significantly};
  \end{tikzpicture}
  \caption{Choice of between the H and \Vrep{} for common operations}
  \label{op}
\end{figure}

\subsection{Plotting}

Polyhedra implements 2D and 3D plotting using \jlpkg{RecipesBase} and \jlpkg{GeometryBasics} respectively.
These library-agnostic interfaces allow
visualizing polyhedra using a variety of options including \jlpkg{MeshCat}, \jlpkg{Makie}~\cite{DanischKrumbiegel2021} and \jlpkg{Plots}~\cite{PlotsJL}.

As shown in \cref{op}, 3D plotting requires both the H and \Vrep{}.
For each half-space, the vertices incident to that half-space are selected.
As these vertices lie on a half-space, triangulating the face is now a planar polyhedral computation problem which is then solved for each face.

\subsection{Chebyshev center}

The Chebyshev center of a hyperrectangle is only unique
if it is a hypercube.
%It is often convenient to obtain the ``most centered'' point of the set of Chebyshev centers.
%Fortunately
%the set of Chebyshev centers
%this set
%is always a polyhedron of dimension than the original set.
\jlpkg{Polyhedra} computes
the unique \emph{proper Chebyshev center} using
the sequence of linear programs
detailed in \cite[Section~4.5.1]{legat2020set},
%this provides a recursive way to compute

\section*{Acknowledgments}

We would like to thank the full list of contributors available \href{https://github.com/JuliaPolyhedra/Polyhedra.jl/graphs/contributors}{on Github}.

The author was supported by
an FNRS Research fellowship,
a Belgian American Educational Foundation (BAEF) postdoctoral fellowship,
the National Science Foundation (NSF) under Grant No. OAC-1835443 and
the European Commission under ERC Grant No. 864017 and ERC Adv. Grant No. 885682.


\input{bib.tex}

\end{document}

% Inspired by the International Journal of Computer Applications template
